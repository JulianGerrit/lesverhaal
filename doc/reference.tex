\documentclass{article}
\usepackage{hyperref}
\begin{document}
\noindent\textbf{\Large Lesverhaal Script Syntax Reference}

\tableofcontents

\tableofcontents

\section{About This Document}
This document describes the syntax of scripts written for Lesverhaal.
Lesverhaal is a visual novel engine that is designed to be used in a classroom
setting. For the player it is similar to RenPy (although it is limited in terms
of presentation). The features that make it useful for teaching, are that
besides multiple choice it supports open questions, and that there is a teacher
interface which allows the teacher to review the student's answers.

\section{Directory Structure}
When starting lesverhaal, there are three locations that are defined using
commandline arguments.
\begin{enumerate}
	\item The data location where lesverhaal reads all the information
		from, by default the current directory. 

	\item The content directory, where image and sound files are stored.
		These are never read by lesverhaal, but only by the player's
		browser. As such, this is not a path from the current
		directory, but an internet address. By default this is
		``content'', which means that if lesverhaal is accessed at
		http://example.com/lesverhaal, the content is searched at
		http://example.com/lesverhaal/content/. It is possible to place
		the content on a different server.

	\item The sandbox directory. This is similar to the content directory,
		but for files uploaded by players using the sandbox interface.
		However, these cannot be at a different server, because
		lesverhaal needs to be able to access those files.
\end{enumerate}

\subsection{The Data Structure}
The data directory contains:
\begin{enumerate}
	\item The startup.py script, which is run to initialize the environment
		of any game.

	\item A users directory. This directory contains one subdirectory per
		group of students, and a directory named ``admin''. All those
		directories contain files with player information. Those files
		have names which are the username of the player converted to
		lowercase.

		To create a new user, it is enough to simply create an empty
		file with the correct name. At login, all information will be
		filled in. This information can later be edited. In particular,
		the password can be removed if it was forgotten.  When there is
		not password in the file, whatever the player enters as the
		password will be accepted and recorded.

		All group directories except admin contain a directory named
		Content (the uppercase C ensures it cannot be mistaken for a
		player). The Content directory contains chapter subdirectories,
		which contain the scripts that this document is mostly about.
		All script files must have the extension ``.script''.
\end{enumerate}

\subsection{The Content Directory}
The content directory contains a subdirectory named ``common'', which contains
common resources to all games, and one subdirectory per chapter, with the same
name as the chapter subdirectory in the group's Content directory. In that
chapter directory, there is a subdirectory for each script, with the same name
as the script file, but without the ``.script'' extension. The content files
for that game are in that subdirectory.

\subsection{The Sandbox Directory}
In the sandbox content directory is a subdirectory for each group and a user
directory under each group directory, with the same name as the group and the
user respectively. All sandbox files are located in that directory, there is no
separate directory for each script. The sandbox scripts are also in that same
directory.

\subsection{Common Resources}
Normally, there is some common world where all (or at least most) games take
place. It would be wrong to place a copy of the resources for those games in
every game directory. Instead, they should be placed in the common directory.
When a script references a file and the address starts with ``common/'', it
will produce a link to the common resources directory. Otherwise, it will
produce a link to the game's data directory (regular or sandbox).

\subsection{Character Data}
Most images in the game are character images. Those are stored in a directory
per character. In that directory are two or more files. When the character is
shown, the ``default'' image is used. When they talk, the ``side'' image is
used beside the text. When they talk with an emotion, the name of the emotion
is used as the filename for the image that is shown. The side image cannot
change.

\section{File Syntax}
The script consists of lines with one command each. Some commands require more
than one line. There are two situations where this occurs:
\begin{enumerate}
	\item Commands which contain sub-commands, such as if and while. Those
		commands end with a colon (:) and the subcommands are indented
		with whitespace. The amount and type of whitespace that is used
		for the indentation is up to the author of the game, but it is
		required to remain the same for the entire block of
		subcommands.

	\item Multi-line strings. These are started with triple quotes, and
		they also end with triple quotes. If the quotes are indented,
		the content must also be indented and the indentation is not
		part of the string. Any extra indentation \textit{is} part of
		the string.
\end{enumerate}

\subsection{Regular Commands}
Empty lines and lines containing only whitespace are ignored.

\subsubsection{\#}
A line starting with \verb-#- is treated as a comment and is ignored.

\subsubsection{\#\#\#}
A line starting with \verb-###- starts a block comment. This extends until a
line that ends with the same marker.

\subsubsection{Quoted Text}
A line of text in quotes is used to define a question to the user. It is
normally followed by a command to define the question type. If multiple lines
of text are given in sequence, they are concatenated and treated as a single
text block.

Some replacements are done on the text. See the section about text for more
information.

\subsubsection{Quoted Text Blocks}
A multi-line block of text can be enclosed in triple quotes. Its meaning is
identical to multiple single lines of text. Both three single and three double
quotes are allowed. The quotes at the start and at the end of the block must be
the same.

\subsubsection{\$}
Run a Python command. If multiple \$-commands are given in sequence, they are
combined to a single Python statement. See the section about Python for more
information.

\subsubsection{id emotion: text}
Using the id of a previously defined character, make the character say
something. If an emotion is specified and the character is currently shown, the
image is changed to that emotion. At the next command, the image is changed
back to the emotion that was last specified in show, or ``default'' if none was
specified.

If the text is omitted, an indented multi-line text block is expected to follow
this command. If the text is omitted and there is no indented block, the
character will say an empty message.

Text replacement is performed on the text. See the section about text for more
information.

\subsubsection{answer type value}
This sets the style for the last answer in the teacher's interface. The type
must be a css attribute. If it is omitted the default is ``background''. Note
that in that case the value cannot contain whitespace.

The value is the value for the css attribute. This is usually a color, which
can be the result of a Python expression.

\subsubsection{break}
This command is only allowed inside a while command. It will cause execution to
ignore the rest of the while loop and immediately continue execution at the
command following the while command.

\subsubsection{character id resource name}
This defines a new character. The id must not be used for any other character.
The resource is the name of the content directory which contains the images for
the character, plus the extension that the image files have. For example, if
``Anja.svg'' is given as the resource for the character, the files that are
used are ``Anja/default.svg'' and ``Anja/side.svg'' (and similarly for the
emotion files).

The name may contain spaces. Also, text replacement is performed on the name.
For example, a name of \verb-${name}- creates a character with the player's
name. See the section about text for more information.

\subsubsection{continue}
This command is only allowed inside a while command. It will cause execution to
ignore the rest of the while loop and immediately retry the while expression.
If it passes, the loop restarts. If it doesn't, the loop ends and execution
resumes at the command following the wile command.

\subsubsection{goto id}
Jump to a label defined elsewhere in the file. If the id starts with a ``.'',
the last defined regular label's name is prefixed to it. See the label command
for more information about goto.

\subsubsection{hide id with transition}
Remove a character's image from screen. If a transition is given, it is used.
The only transitions that are allowed for hide are ``moveoutleft'' and
``moveoutright''.

\subsubsection{if expression:, elif: expression and else:}
Run the indented block that follows if the Python expression evaluates to a
True value.

If present, the elif command must be the first command at the same indentation
level after the if (or other elif) command. Its meaning is identical to the if
command, but it will only run if all previous expressions did not evaluate to a
True value.

If present, the else command must be the first command at the same indentation
level after the if or last elif command. The indented block that follows will
only run if all previous expressions did not evaluate to a True value.

See the section about Python for more information about the expression
evaluation.

\subsubsection{label id}
Defines a label with the given id at this point in the file. Labels can be used
to jump to with goto.

Using labels and goto is only recommended for very simple flow control. In most
cases, it is better to use if and while statements.

Labels are only allowed at top level; they must not occur inside indented
blocks.

If a label name starts with ``.'', it prefixes it with the last defined regular
label. This allows labels that are only used within a limited block to have
names that are not unique throughout the file. For example, if a label named
``.leave'' is defined after a label named ``home'', the new label is actually
named ``home.leave''.

\subsubsection{music filename}
Start playing music, or if the filename is omitted, stop playing music.

When the file finishes playing, it is automatically restarted.

\subsubsection{say name: text}
Present speech to the player.

This is similar to a character talking. However, the name does not need to be
defined for a character. Because there is no link to a character, it is not
possible to implicitly change the emotion of a character with this command. Use
show before and after say for that. There is also no side image for the speech.
To get a side image, use the say function from a Python command. See the
section about Python for more information.

Similar to a talking character, if the text is omitted and a multi-line block
follows this command, it is used as the text.

Text replacement is performed both on the name and on the text. See the section
about text for more information.

\subsubsection{scene image}
Set the background of the screen to the given image.

\subsubsection{show id emotion at position with transition}
Show a character on screen, or change the position or emotion of a character
that is already on screen.

If an emotion is given, it is used for the character's image.

If a position is not given, it is not changed if the character is already on
screen, otherwise ``center'' is used.

If a transition is given, it is used. Options are:
\begin{enumerate}
	\item move: this moves the character from one place to another. The
		character must already be on the screen.
	\item dissolve: the character appears in place by increasing opacity.
		The character must not be on screen yet.
	\item moveinleft: the character is moved into the screen from the left
		side.  The character must not be on screen yet.
	\item moveinright: the character is moved into the screen from the
		right side.  The character must not be on screen yet.
	\item moveoutleft: the character is moved out of the screen to the left
		side.  The character must already be on screen.
	\item moveoutright: the character is moved out of the screen to the
		right side.  The character must already be on screen.
\end{enumerate}

\subsubsection{sound filename}
Play a sound. When the file finishes playing, it is not restarted. If no
filename is given, stop playing the current sound.

\subsubsection{video filename}
Show a video.

\subsubsection{while expression:}
Run the indented block that follows zero or more times. If the Python
expression evaluates to a True value, the block is executed and the expression
is evaluated again. This repeats until the expression does not evaluate to a
True value.

The break and continue commands manipulate the flow in a while loop.

See the section about Python for more information about the expression.

\subsubsection{with transition:}
Move multile characters at the same time.

This command is followed by a block of show and hide commands. The transition
(see the show command for details) is used for all of them, and they are all
moving at the same time. If a transition is specified for a show or hide
command, it takes precedence. It does not change the fact that the characters
move simultaneously.

\subsection{Question Commands}
There are may ways to ask a question in lesverhaal. Some look different to the
player. All are different for the script author.

All question commands can be preceded by a text. This text is shown to the
player, so they know what they should be answering.

Every question gets an id. These ids must be unique in the script file.
Internally the id is used to store the answers. A duplicate id will allow the
player to repeat an answer to a different question, for example.

If the id starts with an underscore (\_), the answer is not saved. Because of
this, those ids do not need to be unique.

In all cases, a Python variable is created with the id as its name and the
answer as its value. How the answer is encoded depends on the question type.

The different question types and their properties are described below.

\subsubsection{number id}
The player inputs a single line of text. It is converted to a number. If the
player did not enter a number, the value is NaN.

Both a decimal point and a decimal comma are accepted as decimal separator.
Thousands separators are not allowed.

\subsubsection{unit id}
The player inputs a single line of text, which is interpreted as a number and a
unit. The value is a sequence of two values. If there is no number, the first
value is None and the second value is the entered answer. Otherwise, the first
value is the number and the second value is the unit.

\subsubsection{short id}
The player inputs a single line of text, which is stored as a string.

\subsubsection{long id}
The player inputs a block of text, which is stored as a string.

\subsubsection{longnumber id, longunit id and longshort id}
The player inputs a block of text and an additional single line. The single
line is parsed as a number, a number with a unit or a short string
respectively. The value is stored as a sequence of two items, the first being
the parsed value and the second a string from the block of text.

Note that a longunit contains a sequence in a sequence, for example:
\verb-[[10,'kg'], '5+5=10']-

\subsubsection{choice id and option text}
The players is presented with multiple buttons, and clicks one.

The choice command must be followed by one or more options. Text replacement is
performed on the option text, see the section about text for more information.

The value that is stored is the index of the selected option. Note that
counting starts at 0.

\subsubsection{longchoice id}
Similar to the other long questions, this is a choice question plus a block of
text. The stored value is a sequence of the index of the selected option and
the text as a string.

Like with the choice command, the options must be given immediately after the
longchoice command and text replacement is performed on the option texts. See
the section about text for more information.

\subsubsection{hidden id expression}
Special question type that does not ask a question to the user. Instead it
evaluates the expression and stores the value with the given id as if the
player gave that answer.  This is useful for the teacher interface. For
example, the game can keep a score and use a hidden question to store the
player's result.

\section{Text Replacement}
There are many texts in the script where text replacement is performed. This
consists of two things:
\begin{enumerate}
	\item If ``FILE://'' appears in the text, it is replaced with a link to
		the game's content directory. So ``FILE://image.png'' will
		point to the image in the game's content. ``FILE://common/'' is
		instead replaced with a link to the common resources.

		Note that this must not be used when filenames are expected,
		such as the argument to the scene command, or a character
		resource definition. No text replacement is performed on those
		fields, and they are automatically converted into links.

	\item A Python expression enclosed in ``\verb-${...}-'' is evaluated
		and replaced with the result. Note that the expression cannot
		contain the ``\verb-}-'' character. If this is needed, use the
		\$ command to evaluate the expression beforehand, then use the
		variable name to get its value.

		See the section about Python for more information about Python
		expressions.
\end{enumerate}

Furthermore, for speech and questions (but not for names), the text is passed
through markdown. This allows the author a lot of freedom to create the look
that they want.

\section{Python}
Every game that is started contains a Python environment. This environment is
used when running Python statements and when evaluating expressions, including
text replacements. When characters are created, a variable with their id as its
name is created. This variable can be used to manipulate the character from the
script. When questions are answered, the answers are stored in a variable as
well.

The initial variables in the Python environment are:
\begin{enumerate}
	\item answer(key, value): similar to the answer command. Also, with
		only one argument given, the key defaults to ``background''.
	\item goto(label): similar to using the goto command.
	\item text(message): similar to using a quoted string as a command.
	\item character(tag, name, url): similar to the character command. Note
		that the arguments are in a different order. Also, a character
		created like this can only speak from Python due to parsing
		limitations.
	\item say(name, text, image = None): similar to the say command, but
		with the option to use a side image.
	\item show(tag, mod = None, at = None, transition = None): similar to
		the show command. tag is the character's id, mod is the
		emotion, at is the location and transition is the transition.
	\item hide(tag, at = 'center', transition = None): similar to the hide
		command.
	\item question(qtype, last\_answer = None, params = ()): Ask a
		question. last\_answer is given to the player as an option to
		repeat it. params is a sequence of options for a choice or
		longchoice question.
	\item user: a dict with user information, including name and group.
		This is not used much, because those two are the most useful
		and they are also stored in their own variable.
	\item name: the player's name.
	\item group: the player's group.
	\item self: the Connection object. This is mostly for internal use. It
		has one useful member: answers, which contains previous answers
		by the player.
\end{enumerate}

\section{Security Considerations}
Lesverhaal runs scripts in a relatively closed environment, which makes it near
impossible that a malfunctioning script will cause damage to the system.
However, there is no protection against malicious scripts. The script can
execute any code on the system that the user running lesverhaal can. Because of
this, access to the sandbox must be explicitly enabled in the user's file, and
this should only be done for trusted users.

If scripts should be written by untrusted users, the entire program must be run
in a protected environment, such as a virtual server.

\section{Moving Images}
Lesverhaal does not support images that keep moving while waiting for user
input. However, it does support svg images, and the svg format supports simple
animations. When an svg file which includes an animation is displayed, the
animation is shown.

\end{document}
